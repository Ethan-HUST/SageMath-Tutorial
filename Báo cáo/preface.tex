\chapter*{Lời nói đầu}
\addcontentsline{toc}{chapter}{Lời nói đầu}

Trong xu hướng chuyển đổi số giáo dục và tăng cường ứng dụng công nghệ trong giảng dạy – học tập, việc sử dụng các phần mềm hỗ trợ tính toán và trực quan hóa Toán học trở nên quan trọng hơn bao giờ hết. SageMath là một công cụ mã nguồn mở mạnh mẽ, không chỉ giúp người học giải quyết các bài toán phức tạp mà còn giúp hình thành tư duy toán học hiện đại, kết hợp lập trình và tư duy phân tích.

Tài liệu này được biên soạn nhằm mục tiêu giúp sinh viên nói riêng và người dùng nói chung tiếp cận SageMath từ cơ bản đến nâng cao, đồng thời cung cấp các ví dụ ứng dụng sát với nội dung của các học phần Toán đại cương tại Đại học Bách Khoa Hà Nội, bao gồm: Giải tích 1, Giải tích 2, Giải tích 3, Đại số tuyến tính và Xác suất thống kê.

Cấu trúc tài liệu được chia thành các chương rõ ràng, từ giới thiệu, hướng dẫn sử dụng phần mềm, các tính năng chính, các tính năng mở rộng đến phần ứng dụng thực tiễn trong từng học phần. Bên cạnh đó, tài liệu cũng cung cấp các phụ lục hữu ích như hướng dẫn khắc phục lỗi cài đặt và danh mục tài liệu tham khảo.

\vspace{1em}

Trong quá trình thực hiện tài liệu này, em đã nhận được sự giúp đỡ và đóng góp ý kiến quý báu từ nhiều thầy cô, bạn bè và đồng nghiệp. Đặc biệt, em xin gửi lời cảm ơn sâu sắc đến thầy Lê Văn Tứ, giảng viên hướng dẫn, người đã tận tình chỉ bảo, định hướng và hỗ trợ em trong suốt quá trình nghiên cứu và biên soạn tài liệu. 

Em cũng xin chân thành cảm ơn các bạn sinh viên trong lớp đã chia sẻ kinh nghiệm sử dụng SageMath, góp phần giúp em hoàn thiện nội dung. Sự động viên và góp ý của mọi người là nguồn động lực to lớn giúp em hoàn thành đồ án này.

Mặc dù đã rất cố gắng nhưng do thời gian hạn chế và kiến thức còn nhiều thiếu sót, tài liệu khó tránh khỏi những hạn chế và sai sót. Em rất mong nhận được sự góp ý từ thầy cô và bạn đọc để tài liệu được hoàn thiện hơn.

\vspace{1em}
\begin{flushright}
	Nguyễn Trung Kiên\\
	Email: \texttt{kien.nt227180@sis.hust.edu.vn}\\
	Hà Nội, tháng 6 năm 2025
\end{flushright}
