\section{Bối cảnh và lý do nghiên cứu}

Trong bối cảnh cuộc cách mạng công nghiệp 4.0, việc tích hợp công nghệ vào giảng dạy và nghiên cứu Toán học ngày càng trở nên cần thiết. Các phần mềm tính toán như SageMath đã mở ra những hướng đi mới, giúp sinh viên và giảng viên tiếp cận với Toán học một cách trực quan, chính xác và hiệu quả hơn.

SageMath là một phần mềm mã nguồn mở mạnh mẽ, được phát triển nhằm mục tiêu cung cấp một nền tảng tính toán toàn diện, kết hợp nhiều hệ thống toán học khác nhau như Maxima, NumPy, SciPy, matplotlib, SymPy, GAP, FLINT, R,... Với triết lý \textit{"Free Open-Source Mathematics Software"} (Phần mềm Toán học Mã nguồn mở Miễn phí), SageMath hướng đến việc thay thế các phần mềm thương mại như MATLAB, Mathematica và Maple trong nhiều ứng dụng.

\section{Mục tiêu của đề tài}

Mục tiêu của đề tài này là:

\begin{itemize}
	\item Giới thiệu tổng quan về SageMath và các tính năng chính của phần mềm.
	\item Hướng dẫn cài đặt, làm quen với giao diện và cách sử dụng cơ bản.
	\item Trình bày các ứng dụng cụ thể của SageMath trong các học phần Toán học cương tại Đại học Bách Khoa Hà Nội.
	\item Cung cấp tài nguyên học tập, ví dụ minh hoạ và hướng dẫn xử lý các lỗi thường gặp khi sử dụng phần mềm.
\end{itemize}

\newpage
\section{Đối tượng sử dụng}

Tài liệu này được thiết kế dành cho:

\begin{itemize}
	\item Sinh viên các ngành kỹ thuật, công nghệ thông tin, toán ứng dụng, kinh tế... đang học các học phần Toán học cơ bản.
	\item Giảng viên, nhà nghiên cứu muốn tích hợp công cụ tính toán vào bài giảng và nghiên cứu khoa học.
	\item Người tự học, đam mê khám phá và ứng dụng Toán học với công cụ tính toán hiện đại.
\end{itemize}

\section{Phạm vi và cấu trúc tài liệu}

Tài liệu được chia thành các chương nhằm tiếp cận từ cơ bản đến nâng cao, với trọng tâm là các ví dụ thực hành có thể chạy trực tiếp trên SageMath. Nội dung trải dài từ hướng dẫn sử dụng, tính năng chính, ứng dụng trong từng học phần đến các tiện ích mở rộng và giải pháp khắc phục lỗi trong quá trình học tập và nghiên cứu.

Tài liệu không đi sâu vào thuật toán, cách xây dựng các hàm toán học đã có của phần mềm mà tập trung vào tính ứng dụng thực tiễn trong giảng dạy và học tập các môn Toán đại cương trên cấp bậc Đại học.
