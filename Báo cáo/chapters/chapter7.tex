\section{Tổng kết nội dung nghiên cứu}

Trong quá trình thực hiện đồ án, em đã nghiên cứu và triển khai việc sử dụng SageMath vào giải quyết các bài toán toán học thuộc nhiều môn học khác nhau trong các học phần Toán đại cương tại Đại học Bách Khoa Hà Nội. 

Thông qua việc khai thác các tính năng cốt lõi của SageMath, em đã đạt được những kết quả đáng khích lệ như sau:
\begin{itemize}
	\item Tìm hiểu và ứng dụng các phép toán cơ bản và nâng cao trong SageMath, bao gồm tính toán đại số, giải tích, tổ hợp, thống kê và các phép toán số học.
	\item Sử dụng các hàm hỗ trợ mạnh mẽ của SageMath để giải quyết các bài toán thực tiễn như tính giới hạn, tích phân, đạo hàm, giải hệ phương trình tuyến tính, chéo hóa ma trận và phân tích Fourier.
	\item Xây dựng các ứng dụng tính toán thực tiễn trong các học phần Toán đại cương, từ các bài toán đại số trừu tượng đến các phép tính trong giải tích số và toán rời rạc.
	\item Tạo các đồ thị trực quan, minh họa các khái niệm toán học thông qua các hàm vẽ đồ thị 2D, 3D, biểu đồ phân phối xác suất, biểu đồ hàm số và các hình học không gian.
	\item Thiết kế SageMath Notebook giúp thầy cô và các bạn sinh viên tương tác và kiểm tra trực tiếp các ví dụ, bài tập, đồng thời tích hợp các thư viện Python để mở rộng khả năng tính toán.
\end{itemize}

Kết quả của đồ án đã chứng minh rằng SageMath là một công cụ mạnh mẽ trong việc hỗ trợ giảng dạy và nghiên cứu toán học, đặc biệt là trong các học phần Toán đại cương tại các trường đại học. Việc tích hợp SageMath vào quá trình học tập giúp sinh viên rèn luyện kỹ năng tính toán, tư duy lập trình, cũng như khả năng phân tích và trực quan hóa dữ liệu toán học.

Tuy nhiên, trong quá trình thực hiện, em cũng nhận thấy một số hạn chế cần khắc phục, sẽ được trình bày trong các phần tiếp theo. Bên cạnh đó, đồ án cũng mở ra những hướng nghiên cứu mới nhằm cải thiện việc ứng dụng SageMath trong giáo dục và nghiên cứu khoa học.

\section{Những khó khăn và hạn chế của SageMath}

Trong quá trình tìm hiểu và sử dụng SageMath, em đã gặp phải một số khó khăn và hạn chế, cả về mặt kỹ thuật lẫn trong việc ứng dụng vào các bài toán thực tiễn. Dưới đây là một số vấn đề nổi bật mà em nhận thấy:

\subsection{Khó khăn trong cài đặt và cấu hình}

Việc cài đặt SageMath không phải lúc nào cũng đơn giản, đặc biệt trên các hệ điều hành như Windows. Một số khó khăn em gặp phải bao gồm:
\begin{itemize}
	\item SageMath yêu cầu dung lượng lớn và tài nguyên hệ thống tương đối cao.
	\item Việc tích hợp với các công cụ như Jupyter Notebook đôi khi gặp lỗi phiên bản và xung đột thư viện.
	\item Cài đặt SageMath trên Windows thường đòi hỏi sử dụng WSL (Windows Subsystem for Linux), gây khó khăn cho những người không quen với môi trường Linux.
\end{itemize}

\subsection{Hạn chế về khả năng tính toán và thư viện}

Mặc dù SageMath tích hợp nhiều thư viện mạnh mẽ, nhưng trong một số trường hợp đặc biệt, em nhận thấy có những hạn chế như sau:
\begin{itemize}
	\item Một số phép toán phức tạp như tính giới hạn, tích phân đa biến hoặc giải phương trình vi phân đôi khi không cho kết quả chính xác hoặc không hỗ trợ trực tiếp.
	\item Các hàm thống kê và phân phối xác suất chưa phong phú so với các phần mềm chuyên dụng như R hoặc MATLAB.
	\item Việc xử lý dữ liệu lớn đôi khi không hiệu quả do thiếu các hàm tối ưu hóa chuyên dụng.
\end{itemize}

\subsection{Hạn chế trong việc trực quan hóa}

Mặc dù SageMath hỗ trợ vẽ đồ thị 2D và 3D, nhưng khi so sánh với các phần mềm như Mathematica hay MATLAB, khả năng trực quan hóa còn hạn chế:
\begin{itemize}
	\item Việc tùy chỉnh đồ thị phức tạp đòi hỏi nhiều mã lệnh và không thân thiện với người mới bắt đầu.
	\item Các đồ thị 3D khi vẽ trong môi trường notebook có thể bị chậm và khó thao tác.
	\item Việc kết hợp nhiều đồ thị trên cùng một hệ trục không dễ dàng như trong các phần mềm chuyên nghiệp.
\end{itemize}

\subsection{Khó khăn trong việc học và sử dụng}

SageMath là một hệ thống mạnh mẽ nhưng không dễ tiếp cận đối với sinh viên mới làm quen với lập trình:
\begin{itemize}
	\item Cú pháp của SageMath đôi khi gây nhầm lẫn vì nó kết hợp nhiều thư viện từ các ngôn ngữ khác nhau.
	\item Tài liệu hướng dẫn chính thức chưa thật sự đầy đủ, đặc biệt đối với các chức năng nâng cao.
	\item Việc tìm kiếm giải pháp cho các lỗi gặp phải trong quá trình sử dụng cũng là một thách thức, do cộng đồng người dùng không lớn bằng các phần mềm thương mại.
\end{itemize}

\section{Định hướng phát triển và nghiên cứu trong tương lai}

Trong quá trình thực hiện đồ án, em nhận thấy SageMath là một công cụ mạnh mẽ nhưng vẫn còn nhiều tiềm năng chưa được khai thác hết. Để nâng cao hiệu quả sử dụng, em xin đề xuất một số định hướng phát triển trong tương lai như sau:

\subsection{Tích hợp AI để chuyển ngôn ngữ tự nhiên thành mã SageMath}

Một trong những khó khăn lớn nhất đối với người dùng SageMath là phải thành thạo cú pháp và các hàm toán học trong phần mềm. Điều này đòi hỏi thời gian học tập và nắm vững ngôn ngữ lập trình của SageMath. 

Để khắc phục vấn đề này, em đề xuất sử dụng trí tuệ nhân tạo (AI) để tự động chuyển đổi ngôn ngữ tự nhiên thành mã SageMath. Mô hình AI sẽ nhận đầu vào là câu lệnh toán học từ người dùng bằng ngôn ngữ tự nhiên, sau đó chuyển đổi thành mã SageMath, thực thi lệnh và trả về kết quả. Điều này sẽ giúp người dùng giải toán mà không cần học SageMath, nâng cao trải nghiệm sử dụng phần mềm, đặc biệt đối với sinh viên và người dùng không chuyên.

\subsection{Phát triển giao diện thân thiện người dùng}

Một trong những yếu tố quan trọng giúp SageMath tiếp cận rộng rãi hơn là việc cải thiện giao diện sử dụng. Hiện tại, giao diện của SageMath chủ yếu dựa trên dòng lệnh hoặc Jupyter Notebook, gây khó khăn cho người mới bắt đầu. 

Trong tương lai, cần phát triển một ứng dụng trực quan hơn, có giao diện đồ họa giống như các phần mềm thương mại khác như Mathematica hay MATLAB. Giao diện này sẽ hỗ trợ:
\begin{itemize}
	\item Nhập công thức toán học trực tiếp bằng cách kéo thả các biểu thức.
	\item Hiển thị kết quả ngay lập tức mà không cần chạy mã lệnh.
	\item Tích hợp các tính năng vẽ đồ thị, giải phương trình và biểu diễn ma trận với các nút bấm trực quan.
\end{itemize}

\subsection{Xây dựng thư viện bài tập và tài liệu hỗ trợ học tập}

Để hỗ trợ sinh viên trong quá trình học tập và nghiên cứu, em đề xuất xây dựng một thư viện bài tập và hướng dẫn giải chi tiết bằng SageMath. Thư viện này sẽ bao gồm:
\begin{itemize}
	\item Các bài tập từ cơ bản đến nâng cao thuộc các học phần Toán đại cương.
	\item Ví dụ minh họa và mã SageMath tương ứng để sinh viên dễ dàng tiếp cận.
	\item Tài liệu giải thích các tính năng quan trọng của SageMath kèm theo ví dụ thực hành.
\end{itemize}

\subsection{Tăng cường khả năng tính toán thống kê và xử lý dữ liệu lớn}

Mặc dù SageMath đã tích hợp các thư viện thống kê và tính toán số học như NumPy, SciPy, nhưng vẫn chưa đáp ứng được nhu cầu xử lý dữ liệu lớn và phân tích thống kê phức tạp. Trong tương lai, cần:
\begin{itemize}
	\item Tích hợp chặt chẽ hơn với các thư viện chuyên dụng như Pandas và R.
	\item Tối ưu hóa hiệu năng để tính toán nhanh hơn khi xử lý dữ liệu lớn.
	\item Cải thiện các hàm thống kê và phân tích dữ liệu để cạnh tranh với các phần mềm chuyên dụng.
\end{itemize}

\subsection{Tăng cường cộng đồng người dùng và tài liệu tham khảo}

Một trong những hạn chế hiện tại của SageMath là cộng đồng người dùng chưa thực sự mạnh như các phần mềm thương mại khác. Để phát triển bền vững, cần:
\begin{itemize}
	\item Khuyến khích người dùng tham gia đóng góp vào mã nguồn và phát triển các gói mở rộng.
	\item Tạo diễn đàn trao đổi kinh nghiệm, giải đáp thắc mắc và chia sẻ tài liệu hướng dẫn.
	\item Xây dựng khóa học và video hướng dẫn sử dụng SageMath từ cơ bản đến nâng cao.
\end{itemize}

\subsection{Kết luận}

Với những định hướng trên, em hy vọng rằng SageMath sẽ ngày càng hoàn thiện và trở thành công cụ toán học mạnh mẽ, tiện dụng hơn cho sinh viên và các nhà nghiên cứu. Việc kết hợp giữa trí tuệ nhân tạo, giao diện thân thiện và cộng đồng người dùng mạnh mẽ sẽ giúp SageMath đạt được mục tiêu phổ cập trong giảng dạy và nghiên cứu toán học.
